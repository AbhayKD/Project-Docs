%This is a very basic  BE PROJECT Synopsis template.


%############################################# 
%#########Author :  PROJECT Synopsis###########
%#########COMPUTER ENGINEERING############

%\title{SYNOPSIS}

\documentclass[oneside,a4paper,12pt]{article}

\usepackage{amsmath}
\usepackage{amssymb}
\usepackage{mathptmx}
\usepackage{amsfonts}

\usepackage{color}
\usepackage{graphicx} % Required for the inclusion of images
\setlength\parindent{0pt} % Removes all indentation from paragraphs
\usepackage{times} 
\usepackage{fancyhdr}
\pagestyle{fancy}
\fancyhf{}
\fancyhead[C]{SYNOPSIS}%Header of Document
\makeatletter
\let\ps@IEEEtitlepagestyle\ps@fancy
\makeatother


%\usepackage{showframe}
%\hoffset = 8.9436619718309859154929577464789pt
%\voffset = 13.028169014084507042253521126761pt
\begin{document}
%\maketitle 
\section{Group Id}
Mention Group ID

\section{Project Title}
Voice Controlled Personal Assistant and Connecting IOT Devices
\section{ Project Option }
Internal Project

\section{Internal Guide}
Prof. Nitin R. Talhar


\section{Technical Keywords (As per ACM Keywords)}
 {\bfseries Technical Key Words:}      
 \begin{itemize}
 \item 	Special-Purpose and Application-Based System
 \item	Online Information Services
 \item  Natural Language Processing
 \item  Input/Output and Data Communication
 \item  Artificial Intelligence
 \item  Distributed System
 \item  Personal Computing
 \end{itemize}
%Please note ACM Keywords can be found : http://www.acm.org/about/class/ccs98-html \\
%Example is given as
%\begin{enumerate}
%	\item C. Computer Systems Organization 
%	\begin{enumerate}
%		\item C.2 COMPUTER-COMMUNICATION NETWORKS 
%		\begin{enumerate}
%			\item C.2.4 Distributed Systems 
%			\begin{enumerate}
%				\item  Client/server 
%\item Distributed applications
%\item Distributed databases
%\item Network operating systems 
%\item Distributed file systems
%\item Security and reliability issues in distributed applications
%	 		\end{enumerate} 
%		\end{enumerate} 
%	  
%
%	
%	\end{enumerate}
%\end{enumerate}



\section{Problem Statement}
\label{sec:problem}
To develop a hardware system which takes the input through voice commands and perform the various actions and keeps learning the context of these commands to further improvise the responses in future and help humans with day-to-day workforce.


\section{Abstract}
In the Modern Era of fast moving Technology we can do things which we never thought we could do before. But to achieve the accomplish these thought theres a need for a platform which can automate all our task with easy and comfort.\\

So there is a need to develop a voice controlled personal AI having brilliant powers of deduction and the ability to interact with our surroundings just by one of the materialistic form of human interaction, our VOICE. The Hardware device captures the personals audio request through microphone and  processes the request so that the device can respond to the individual using in-built speaker module.For Example, if you ask the device 'what's the weather?' or 'how's traffic?' using its built-in skills, it looks up the weather and returns the response to the customer through connected speaker.\\

\noindent
The platform is open to all and connect all IOT devices in the vicinity to perform the assigned task on go. This feature makes it distinctive from already existing personal AI’s and separates it from the flock. It uses open source software to process natural language, to determine the intent of the query and to perform the action. The platform features include : Turn on the lights, do the mathematics, play your favorite song or ask anything which comes to your mind. Just speak naturally and the platform is there to do your bidding.\\

\noindent
The Platform also a extends an Android Application in which you can add To-Do-list or set a reminder or alarm through the app, push notifications and personalize according to your needs. The device has a lot of native skills and abilities baked in it and more can be added to extend its capabilities.\\

\noindent
There is still a lot of ground to be covered up in the world of automation but the skills the device we are building posses, it can help to build a new generation of voice controlled devices and bring a new  sustaining change in the field of automation.\\

\section{Goals and Objectives}

\begin{itemize}
	\item The main objective of this device is to ease the burden of your work by providing any information you want and help you with daily work.
	\item To expedite you from the pressure of remembering things like meeting, play music on go, solve your queries.
	\item To develop a platform which can interact seamlessly and can have a friendly conversation with you.
	\item To control IOT devices of the surroundings of the device using simple commands.
\end{itemize}

	
\section{Relevant mathematics associated with the Project}
\label{sec:math}
System Description:
\begin{itemize} 
\item Input:	 
\item Output:	 
\item Identify data structures, classes, divide and conquer strategies to exploit distributed/parallel/concurrent processing, constraints. 
\item Functions : Identify Objects, Morphisms, Overloading in functions, Functional relations
\item Mathematical formulation if possible
\item Success Conditions:	 
\item Failure Conditions:		
\end{itemize}


\section{Names of Conferences / Journals where papers can be published}
\begin{itemize}
\item  IEEE/ACM Conference/Journal 1 
\item  Conferences/workshops in IITs
\item  Central Universities or SPPU Conferences 
\item IEEE/ACM Conference/Journal 2 
\end{itemize}


\section{Review of Conference/Journal Papers supporting Project idea}
\label{sec:survey}
   Atleast 10 papers + White papers or web references\\
   Brief literature survey [ Description containing important description of at least 10 papers

\section{Plan of Project Execution}
  Using planner or alike project management tool.
\end{document}